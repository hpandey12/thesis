\definecolor{rwth}{RGB}{0,102,204}			% RWTH blue

\setlength{\parindent}{4ex}
\setlength{\parskip}{1ex}
\evensidemargin0cm	
\oddsidemargin0mm

\setcounter{tocdepth}{2} 						% Depth of table of contents
\setcounter{secnumdepth}{2}
\hypersetup{
    pdftitle={\titel},
    pdfauthor={\autor},
    pdfcreator={\autor},
    pdfsubject={\titel},
    pdfkeywords={\titel},
}

% ======================================================
% ======================================================
% ========= Head & Footer
% ======================================================
% ======================================================

\pagestyle{scrheadings}							% Kopf- und Fu�zeilen mit KOMA-Script
\clearscrheadings								% Bisherige Kopf- und Fu�zeilen l�schen
\clearscrplain
\lohead{\pagemark}								% links, oben, gerade Seite
\lehead{\leftmark}								% links, oben, ungerade Seite
\cohead{}										% mittig, oben, gerade Seite
\cehead{}										% mittig, oben, ungerade Seite
\rohead{\leftmark}								% rechts, oben, gerade Seite								
\rehead{\pagemark}								% rechts, oben, ungerade Seite
\lofoot{}										% links, oben, gerade Seite
\lefoot{}										% links, oben, ungerade Seite
\cofoot{}										% mittig, oben, gerade Seite
\cefoot{}										% mittig, oben, ungerade Seite
\rofoot{}										% rechts, oben, gerade Seite								
\refoot{}										% rechts, oben, ungerade Seite
\setheadsepline{1.2pt}[\color{rwth}]			% Linie unter Kopfzeile


% ======================================================
% ======================================================
% ========= List of symbols & List of abbreviations
% ======================================================
% ======================================================

% interessant: http://www.mrunix.de/forums/showthread.php?t=42536

\newcommand{\abkVZ}[2]{\nomenclature[A_#1]{#1}{#2}}
\newcommand{\frmVZ}[3][2]{\nomenclature[C_#1]{#2}{#3}}

\makenomenclature
\renewcommand{\nomlabelwidth}{3cm}			% Linke Spaltenbreite
\newcommand{\nomnameA}{List of Abbreviations}
\newcommand{\nomnameB}{List of Symbols}
\renewcommand{\nomname}{\nomnameA}
\newcommand{\nomaltname}{\nomnameB}
\newcommand{\nomaltpreamble}{}
\newcommand{\nomaltpostamble}{}
\newcommand{\usetwonomenclatures}{\nomenclature[\switchnomitem]{}{}}
\newcommand{\switchnomitem}{B}
\renewcommand{\nomgroup}[1]{
	\ifthenelse{ % IF
		\equal{#1}{\switchnomitem}
	}{ % THEN
		\switchnomalt
	}{ % ELSE
	}
}

\setlength{\nomitemsep}{-\parsep}
\newcommand{\switchnomalt}{
	\end{thenomenclature}
	\renewcommand{\nomname}{\nomaltname}
	\renewcommand{\nompreamble}{\nomaltpreamble}
	\renewcommand{\nompostamble}{\nomaltpostamble}
	\begin{thenomenclature}
	\addcontentsline{toc}{chapter}{\nomnameB}	
	\markboth{\nomnameB}{\nomnameB}
	\vspace*{-1.8em}
}

% ======================================================
% ======================================================
% ========= Margin next to equations
% ======================================================
% ======================================================

\g@addto@macro\normalsize{%
	\setlength\abovedisplayskip{20pt}
	\setlength\belowdisplayskip{20pt}
	\setlength\abovedisplayshortskip{20pt}
	\setlength\belowdisplayshortskip{20pt}
}

% ======================================================
% ======================================================
% ========= Userdefined commands
% ======================================================
% ======================================================

\newcommand{\charef}[1]{Chapter~\ref{#1}}
\newcommand{\secref}[1]{Section~\ref{#1}}
\newcommand{\figref}[1]{Figure~\ref{#1}}
\newcommand{\tabref}[1]{Table~\ref{#1}}
\newcommand{\equref}[1]{Equation~\eqref{#1}}
